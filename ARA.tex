\documentclass[titlesec]{araproposal}

% This documentclass is based on the article class, so options to article can
% still be used.
%
% There are two new options:
%   titlesec: reduce the spacing around the section titles by loading the
%     titlesec package.
% noparskip: turn off the spacing between paragraphs and
%   revert to the
%     default indent-based paragraph spacing.

\title{Unofficial ARA Proposal Template}

\piauthor{PI Name, Title, Department Name, University Name}

% CoPIs are optional
\copiauthorA{Co-PI Name, Title, Department Name, University Name}

\copiauthorB{Co-PI Name, Title, Department Name, University Name}

\cash{80,000 USD}

\awscredits{0 USD}

% Contact is optional and can be commented out.
\contact{Contact Name, \email{contact@example.com}}
\date{}

\newcommand{\semph}[1]{\textbf{\textit{#1}}} % Strong emphasis
\newcommand{\para}[1]{\textbf{#1.}\ }


\begin{document}

\maketitle

\abstract{This is an \emph{unofficial} \LaTeX{} template for ARA proposals created
based on the official docx template in 2023 \cite{ara}.
Please carefully read the official instructions when using this template.}

\keywords{research, funding, proposal}

\section{Introduction}
\textit{Significance of the research and prior work.}

\section{Methods}
\textit{Your technical approach.}

\section{Expected Results}
\textit{Please include milestones with timeline estimates, such as for datasets, code
releases, technical reports, publications, applications, presentations, etc.}

\section{Funds Needed}
\textit{Please provide justification supporting the designated amounts for cash
funding and AWS Promotional Credits. For cash funding, please do not include
indirect expenses which are not incurred solely for your project. For AWS
Promotional Credits, please list the AWS products you plan to use.}


\section{Additional Information}
\textit{If the CFP you are applying for requests addition information in the “Proposal
requirements” section, please provide the information here.}

The following ones are questions for \textit{Automated Reasoning Call for
Proposals --- Fall 2023}.

\begin{question}
  Does your work target analysis of protocols, code, or configuration? {[\dots]} %Please provide information about your domain and the type of analysis.
\end{question}

\begin{answer}%
  Yes.
\end{answer}

\begin{question}
  What are the current applications of your work? (e.g., libraries, codebases, or industry code).
\end{question}

\begin{answer}%
  Yes.
\end{answer}

\begin{question}%
  What are potential applications of your work to Amazon?
\end{question}

\begin{answer}%
  Yes.
\end{answer}

\begin{question}
  What assumptions are made by your work? [\dots] % If the techniques proposed are sound: What are issues that may invalidate this result?
\end{question}

\begin{answer}%
  Yes.
\end{answer}

\begin{question}
  If your work involves the development and maintenance of a tool [\dots]
  %:
  %(a) Under what license is or will your tool be released?
  %(b) What on-boarding/tutorial material is available?
  %(c) Is your tool actively maintained (i.e., commits within last 3 months)? How many active contributors does your project have?
\end{question}

\begin{answer}%
  Yes.
\end{answer}

\appendix

\newpage
\bibliographystyle{plainurl} 
\bibliography{ref} 

\newpage
\section{CV of PI}
\textit{One-page CV of PI (professional title, experience, etc.), only to include the 5-6 most relevant papers to this proposal. One-page CV of Co-PI (optional).}

\newpage
\section{Previously Funded Project Summary}
\textit{If you or someone on your project team has received cash funding or AWS Promotional Credits directly or indirectly from Amazon within the past 3 years, please answer the questions below. (Include funding from Amazon through your institution or from the NSF-Amazon Fairness in AI funding grant.)}

\begin{itemize}
  \item \textit{Project title:}
  \item \textit{Funding amount and time:}
  \item \textit{Funding source:}
  \item \textit{Who was this funding for?}
  \item \textit{If AWS Promotional Credits were provided, how much is left?}
  \item \textit{Please provide a brief summary of the results, papers, and open-source software enabled:}
\end{itemize}
\end{document}
